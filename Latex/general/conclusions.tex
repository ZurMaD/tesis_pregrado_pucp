%----------------------------------------------------------------------------------------
%	Agradecimientos
%----------------------------------------------------------------------------------------


\chapter*{\centering \large Conclusiones}
\addcontentsline{toc}{chapter}{Agradecimientos}
\markboth{Agradecimientos}{Agradecimientos}

%\justify
%{Escribir agradecimientos aquí}\\[0.5cm]

Se detectaron procesos críticos en la crianza de truchas y se seleccionó el proceso manual que disminuía la producción final generando pérdidas económicas: clasificación y conteo de truchas en la etapa de engorde (15 a 20 centímetros).

Se elaboró la lista de requerimientos según una entrevista con personas dedicadas al cultivo de truchas. Se propuso tres conceptos de solución y se escogió uno de estos bajo un análisis técnico-económico. El concepto de solución óptimo cumple con todos los puntos de la lista de requerimientos. Además, se realizó el diagrama de operaciones necesario para mostrar el funcionamiento de la máquina.

Para el análisis de estabilidad y flotabilidad sobre el agua del concepto de solución óptimo será necesario realizar cálculos y mediciones 

Una estimación simple sin detalle del concepto de solución óptimo muestra un costo menor comparado con el costo de una máquina que se comercializa internacional-mente. Además, el costo de operación disminuye debido a que se reduce el número de operarios de cuatro a solo uno.

El trabajo presentado puede ser extrapolado para obtener conceptos de solución diseñados para otras tallas de truchas.

