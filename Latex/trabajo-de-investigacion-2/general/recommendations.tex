%----------------------------------------------------------------------------------------
%	Agradecimientos
%----------------------------------------------------------------------------------------

\pagestyle{myportland}
\doublespacing
%\pagenumbering{arabic}
\chapter*{\centering \large Recomendaciones}
\addcontentsline{toc}{chapter}{Recomendaciones}
\markboth{Recomendaciones}{Recomendaciones}
\thispagestyle{myportland}

En el presente trabajo, acorde a la complejidad, se plantea un desarrollo simple del diseño frontend de la aplicación móvil, enfocado en los pasos a seguir y las opciones mínimas que se requieren. Además, dicho diseño no fue probado en un grupo de personas a las que se orienta su uso. Lo óptimo es realizar iteraciones de diseño para obtener simplicidad junto con los principios de diseño intuitivo y amigable, es decir, centrado en el humano. Este punto es de proceso contínuo por lo que puede tardar desde meses hasta años: razón suficiente para realizar una simplificación conceptual.

El análisis de algoritmo de procesamiento de imágenes que procesará el conteo y clasificación de truchas se guía sobre indicadores y experimentos realizados por diversos autores. Lo ideal es realizar el trabajo y probar las redes neuronales del estado del arte directamente al contexto, es decir, realizar pruebas en campo, con condiciones reales para elegir una red neuronal de manera óptima.

Con el objetivo de mejorar la segmentación de la trucha se debe experimentar en producción el uso de otros colores de iluminación para brindar un contraste con el pez, así como el posicionamiento de estas dentro de la sección en la que se realiza el procesamiento de imágenes. 

Debe experimentarse sobre la iluminación adecuada acorde a la velocidad de fotogramas requerida, ya que a mayor velocidad de fotogramas se requiere mayor iluminación. Sin embargo, no se halló referencias bibliográficas sobre el impacto de la luz artificial en los órganos de la visión de las truchas.

Las tecnologías que se usan en los algoritmos presentes en este trabajo están enfocadas a la investigación y desarrollo, por lo que la aplicación de estas en producción debe ser rediseñada con el fin de optimizar el rendimiento y consumo de batería.

El diseño de ingeniería tanto en la etapa de investigación como en producción es iterativo contínuo, es decir, siempre se puede obtener un producto con mayor acabado, mejor rendimiento y que cumpla con excelencia los requerimientos. Para llegar a un punto óptimo, en el que se tenga un producto comerciable, debe realizarse iteraciones y observaciones de profesionales en la industria, así como pruebas con los potenciales clientes que usarán el sistema.

Los algoritmos de control empleados cumplen con la misión de ejercer un control simple. Sin embargo, diversos autores recomiendan que estos deben ser robustos al momento de evolucionar el diseño de desarrollo a producción, por lo que lo óptimo sería automatizar el proceso mediante algoritmos de última generación que garanticen seguridad y una respuesta de cambio suave que no pueda dañar los sistemas mecánicos administrados por el sistema.




