%----------------------------------------------------------------------------------------
%	Agradecimientos
%----------------------------------------------------------------------------------------

\pagestyle{myportland}
\doublespacing
%\pagenumbering{arabic}
\chapter*{\centering \large Conclusiones}
\addcontentsline{toc}{chapter}{Conclusiones}
\markboth{Conclusiones}{Conclusiones}
\thispagestyle{myportland}

Se desarrolló el diseño de ingeniería del sistema que clasifica y cuenta truchas de 15 a 20 cm. basado en el diseño conceptual óptimo. Se estima que el sistema puede procesar sin problemas el proceso completo de clasificación de una jaula flotante de 5x5 m. en un día, a diferencia de los 7 días con mano de obra. Este diseño podría ser una alternativa real y permite el desarrollo de un prototipo que pueda resultar en un producto comercial.

Se han cumplido los requerimientos del sistema para que funcione sin problemas técnicos en las condiciones de la laguna de Paucarcocha, conceptos mecánicos, eléctricos y electrónicos han sido diseñados acorde al entorno. Principios de diseño centrado en el humano y no contaminación y durabilidad han sido considerados, los materiales como el acero inoxidable 316 del sistema no representan agentes contaminantes para el entorno de trabajo.

El precio de producción de una máquina es de S/ 17535.62 ($\approx$4900), aunque este precio no es comparable con las máquinas comerciales, nos brinda una idea del costo de producción en lotes que abaratan los costos y podría ser accesible para empresas pequeñas y medianas en el Perú.

El sistema de procesamiento de imágenes para detección y conteo de truchas pueden detectar y registrar medidas del pez en tránsito que pasa frente a la cámara a una velocidad de 2 m/s. Acorde a la bibliografía y a los avances tecnológicos la mejora en este sistema era predecible, y aseguran un mayor rendimiento en las siguientes décadas.
