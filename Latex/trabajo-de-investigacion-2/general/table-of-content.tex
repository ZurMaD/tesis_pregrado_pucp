%----------------------------------------------------------------------------------------
%	Tabla de contenidos
%----------------------------------------------------------------------------------------

%\newpage
%\clearpage{\pagestyle{empty}\cleardoublepage}
%\doublespacing
%\newpage

\newpage
\addcontentsline{toc}{chapter}{\contentsname}
\tableofcontents 
\newpage

\addcontentsline{toc}{chapter}{\listfigurename}
\listoffigures 
\newpage

\addcontentsline{toc}{chapter}{\listtablename}
\listoftables 
\newpage

\addcontentsline{toc}{chapter}{Índice de símbolos}

\begin{center}
	{\Large  \textbf{Índice de símbolos}}
\end{center}

Se utilizan las siguientes normas DIN para la definición y designación de símbolos\\

\begin{mytable}[H]
	%\footnotesize
	\centering
	%\caption{Pines necesarios en el microprocesador.}
	%\label{tab:pines necesarios en el microprocesador}
	\begin{tabular}{ll}
		\multicolumn{1}{c}{Concepto} & \multicolumn{1}{c}{Norma} \\
		Unidades & DIN 1301 \\
		Signos matemáticos & DIN 1302 \\
		Notación para fórmulas en general & SI 2019 \\
		Masa, peso, fuerza de peso, aceleración de caída & DIN 1305 \\
		%Rotación. Hélice. Ángulo & DIN 1312 \\
		Densidad & DIN 1306 \\
		Presión & DIN 1314 \\
		Redondeado de números & DIN 1333 \\
		%Estado normal Volumen normal & DIN 1343 \\
		%Notación en resistencia de materiales & DIN 1350
	\end{tabular}
	%\begin{myflushcenteraftertable}	
	%	Fuente: Elaboración propia.
	%\end{myflushcenteraftertable}
\end{mytable}


%%%%%%%%%%%%%%%%%%%%%%%%%%%%%%%%%%%%%%%%%%%%%%%%%%%%%%%%%%%%

\begin{center}
	\textbf{Unidades de medida según DIN 1301}
\end{center}

\begin{mytable}[H]
	%\footnotesize
	\centering
	%\caption{Pines necesarios en el microprocesador.}
	%\label{tab:pines necesarios en el microprocesador}
	\begin{tabular}{ll}
		\multicolumn{1}{c}{} & \multicolumn{1}{c}{} \\
		Longitud & $m$ (Metro) \\	
		Superficie & $m^2$ (Metro cuadrado) \\
		Volumen & $m^3$ (Metro cúbico) \\
		Ángulo plano & $rad$ (Radiante) \\
		Tiempo & $s$ (segundo) \\
		Frecuencia & $Hz$ (Hertz) \\
		Masa & $Kg$ (Kilogramo) \\
		Temperatura & °$K$ (Grado Kelvin) \\
		Temperatura & °$C$ (Grado Celsius) \\
		Fuerza & $N$ (Newton) \\
		Presión & $N/m^2$ (Newton/metro cuadrado) \\
		Energía, trabajo, cantidad de calor & $J$ (Joule) \\
		Potencia & $W$ (Watt) \\
		Viscosidad dinámica & $N.s/m^2$ \\
		Viscosidad cinemática & $m^2/s$ \\
		Intensidad de corriente & $A$ (Ampere) \\
		Tensión eléctrica & $V$ (Volt) \\
		Resistencia eléctrica & $\Omega$ (ohm) \\
		%Conductividad & $S$ (Siemens) \\
		%Capacidad eléctrica & $F$ (Farad) \\
	\end{tabular}
	%\begin{myflushcenteraftertable}	
	%	Fuente: Elaboración propia.
	%\end{myflushcenteraftertable}
\end{mytable}

%%%%%%%%%%%%%%%%%%%%%%%%%%%%%%%%%%%%%%%%%%%%%%%%%%%%%%%%%%%%
\newpage 
\begin{center}
	\textbf{Símbolos en fórmulas según DIN 1304}
\end{center}

\begin{mytable}[H]
	%\footnotesize
	\centering
	%\caption{Pines necesarios en el microprocesador.}
	%\label{tab:pines necesarios en el microprocesador}
	\begin{tabular}{ll}
		\multicolumn{1}{c}{\quad\quad\quad} & \multicolumn{1}{c}{} \\
		$\alpha, \; \beta, \; \gamma$ & ángulo \\	
		$l$\quad\quad\quad& longitud \\
		$b$ & anchura \\
		$h$ & altura \\
		$r, \; R$ & radio \\
		$d,\; D$ & diámetro \\
		$A$ & Área \\
		$S$ & superficie \\
		$V$ & volumen \\
		$t$ & tiempo \\
		%$\omega$ & Velocidad angular \\
		%$\alpha$ & Aceleración angular \\
		$g$ & aceleración de la gravedad \\
		$T$ & duración del periodo \\
		$F$ & frecuencia \\
		$n$ & número de revoluciones \\
		$m$ & Masa \\
		$\rho$ & densidad \\
		$J$ & momento de inercia \\
		$F$ & fuerza \\
		%$G$ & fuerza de peso \\
		$M$ & momento \\
		$P$ & presión \\
		$\sigma$ & tensión de tracción o compresión \\
		$\tau$ & tensión tangencial, tensión de cortadura \\
		%$E$ & módulo de elasticidad \\
		%$G$ & módulo tangencial \\
		%$I$ & momento de inercia superficial \\
		%$M$ & coeficiente de rozamiento \\
		$\eta$ & viscosidad dinámica \\
		$\nu$ & viscosidad cinemática \\
		%$\gamma$ & tensión superficial \\
		$E$ & energía \\
		$P$ & potencia \\
		$T$ & temperatura Kelvin \\
		$t$ & temperatura Celsius \\
		$A$ & conductividad térmica \\
		%$S$ & entropía \\
		%$s$ & entropía específica \\
		%$H$ & entalpia \\
		$R$ & resistencia eléctrica, resistencia efectiva \\
	\end{tabular}
	%\begin{myflushcenteraftertable}	
	%	Fuente: Elaboración propia.
	%\end{myflushcenteraftertable}
\end{mytable}

%%%%%%%%%%%%%%%%%%%%%%%%%%%%%%%%%%%%%%%%%%%%%%%%%%%%%%%%%%%%

\begin{mytable}[H]
	%\footnotesize
	\centering
	%\caption{Pines necesarios en el microprocesador.}
	%\label{tab:pines necesarios en el microprocesador}
	\begin{tabular}{ll}
		\multicolumn{1}{c}{\quad\quad\quad} & \multicolumn{1}{c}{} \\
		%$h$ \quad\quad\quad& entalpia específica \\	
		%$R$ & constante de gas \\
		%$Q$ & cantidad de carga eléctrica \\
		%$U$ & tensión eléctrica \\
		%$C$ & capacidad eléctrica \\
	\end{tabular}
	%\begin{myflushcenteraftertable}	
	%	Fuente: Elaboración propia.
	%\end{myflushcenteraftertable}
\end{mytable}

\begin{center}
	\textbf{Subíndices en fórmulas según DIN 1304}
\end{center}

\begin{mytable}[H]
	%\footnotesize
	\centering
	%\caption{Pines necesarios en el microprocesador.}
	%\label{tab:pines necesarios en el microprocesador}
	\begin{tabular}{ll}
		\multicolumn{1}{c}{\quad\quad\quad} & \multicolumn{1}{c}{} \\
		$max$  \quad\quad\quad& máximo \\	
		$min$  & mínimo \\
		%$zul^* (adm)$  & admisible \\
		$t$ &  componente tangencial \\
		$r$ & relativo, radial \\
		%$n$ & componente normal \\
		$a^* (e)$ & exterior \\
		$i$ & interior \\
		$0$ & valor característico, valor inicial valor de reposo \\
	\end{tabular}
	%\begin{myflushcenteraftertable}	
	%	Fuente: Elaboración propia.
	%\end{myflushcenteraftertable}
\end{mytable}


%%%%%%%%%%%%%%%%%%%%%%%%%%%%%%%%%%%%%%%%%%%%%%%%%%%%%%%%%%%%

\begin{center}
	\textbf{Notación de resistencia de materiales según DIN 1350}
\end{center}

\begin{mytable}[H]
	%\footnotesize
	\centering
	%\caption{Pines necesarios en el microprocesador.}
	%\label{tab:pines necesarios en el microprocesador}
	\begin{tabular}{ll}
		\multicolumn{1}{c}{\quad\quad\quad} & \multicolumn{1}{c}{} \\
		$\sigma$\quad\quad\quad& tensión normal \\	
		$\tau$ & tensión tangencial admisible \\
		$\sigma_{adm}$ & tensión normal admisible \\
		$\tau_{adm}$ & tensión tangencial admisible \\
		$v^* (F.S)$ & factor de seguridad \\
		$\sigma_{F}$ & tensión en el límite de fluencia \\
		$\sigma_{B}$ & resistencia a la tracción \\
		$E$ & módulo de elasticidad \\
		%$G$ & módulo de elasticidad transversal \\
		$M$ & Momento de una fuerza \\
		$M_{t}$ & momento de torsión \\
		$M_{b}$ & momento de flexión \\
		%$i$ & radio de inercia \\
		%$W$ & momento resistente \\
		%$S$ & momento estático de una superficie \\
		%$J_{p}$ & momento de inercia polar \\
		%$J$ & momento de inercia superficial \\
	\end{tabular}
	%\begin{myflushcenteraftertable}	
	%	Fuente: Elaboración propia.
	%\end{myflushcenteraftertable}
\end{mytable}

%%%%%%%%%%%%%%%%%%%%%%%%%%%%%%%%%%%%%%%%%%%%%%%%%%%%%%%%%%%%

%\begin{center}
%	\textbf{Símbolos de uso particular según DIN 1350}
%\end{center}

%\begin{mytable}[H]
	%\footnotesize
%	\centering
	%\caption{Pines necesarios en el microprocesador.}
	%\label{tab:pines necesarios en el microprocesador}
%	\begin{tabular}{ll}
%		\multicolumn{1}{c}{\quad\quad\quad} & \multicolumn{1}{c}{} \\
%		$c_{1,2}$ \quad\quad\quad& factor de corrección de proporción entre D y d \\	
%		$c_{c}$ & confiabilidad estadística \\
%		$c_{carg}$ & factor de carga \\
%		$c_{s}$ & coeficiente de superficie \\
%		$c_{t}$ & coeficiente de tamaño \\
%		$c_{temp}$ & coeficiente de temperatura \\
%		$R_{t}$ & profundidad de rugosidad \\
%		$R_{z}$ & profundidad promedio de la rugosidad \\
%		$\beta_{k}$ & factor efectivo de concentración de esfuerzos \\
%		$\beta_{k-flexio^{´}n}$ & $\beta_{k}$ para esfuerzos que ocasionan flexión \\
%		$\beta_{k \; (2,0)}$ & $\beta_{k}$ para $D/d =2$ \\
%	\end{tabular}
	%\begin{myflushcenteraftertable}	
	%	Fuente: Elaboración propia.
	%\end{myflushcenteraftertable}
%\end{mytable}

\newpage
