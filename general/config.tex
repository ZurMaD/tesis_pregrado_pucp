\documentclass[12pt,a4paper,oneside]{report} % Formato tesis, 12 tamaño de letra
\usepackage[english,spanish]{babel} % Asignar idioma por defecto
\selectlanguage{spanish} % Idioma por defecto x2
\usepackage[T1]{fontenc} % Aceptar caracteres con tíldes en el pdf generado
\usepackage[UTF8]{ctex}
\usepackage[utf8]{inputenc} % Aceptar caracteres con tíldes en Latex
\usepackage{amsmath} % Para mostrar de forma adecuada las ecuaciones
\usepackage{graphicx} % Para incluir imágenes 
\usepackage{array} % Para incluir imagenes en tablas https://bit.ly/2KXhIlu
\usepackage{float} % Para posicionar imágenes y tablas
\usepackage{longtable} % Para tablas que abarcan más de dos páginas
\usepackage{multicol} % Para poder partir en columnas 
\usepackage[refpages]{gloss} % Para enumerar páginas
\usepackage{anysize} % OBSOLETO, NECESITA SER CAMBIADO
\usepackage{bigstrut} % Para automatizar ingresado de datos en tabla
\usepackage{appendix} % Formato para títulos de los apéndices
\usepackage{lscape} % Modifica margenes y rota la página pero no el enumerado
\usepackage{pdflscape} % Cambiar orientación en el pdf
\usepackage{multirow} % Fácil manejo de filas
\usepackage{listings} % Sirve para que se compile el Latex
\usepackage{color} % Colores de fondo, texto y otros
\usepackage{setspace} % Espacio entre lineas
\usepackage{enumerate} % Estilo del contador de páginas
\usepackage{ragged2e} % Formato de parrafos, justiticar y demás
\usepackage{comment} % Hacer comentarios en latex
\usepackage{pslatex} % Tipo de fuente
\usepackage{apacite} % Citación en APA
\usepackage{fixltx2e} % Corrige bugs de Latex 3
\usepackage{caption} % Objetos flotantes en el documento
\usepackage[top=2.54cm,
			bottom=2.54cm,
			left=2.54cm,
			right=2.54cm,
			headheight=17pt, % as per the warning by fancyhdr
			includehead,includefoot,
			heightrounded, % to avoid spurious underfull messages
			]{geometry} % Márgenes para contenido y líneas de pie y cabecera de página
\usepackage{fancyhdr} % Cabeza y pie de página 
\usepackage{blindtext}
\usepackage[table,xcdraw]{xcolor} % Para tablas generadas en https://www.tablesgenerator.com/
\usepackage{pdfpages} % Cargar imagenes en pdf
\usepackage{datetime} % Fecha
\usepackage{footnote} % Tablas largas y foot note para tablas https://texblog.org/2012/02/03/using-footnote-in-a-table/

\usepackage{url} % Evitar enlaces en bibliografía largos & enlazar enlaces de bibliografía
%\usepackage{hyperref} % Evitar enlaces en bibliografía largos & enlazar enlaces de bibliografía

\fancyhead{}  % Clears all page headers and footers
\setlength{\headheight}{15pt}
\pagestyle{fancy}
%\fancyhead[r]{\leftmark}
%\fancyfoot[c]{PUCP}
%\fancyfoot[r]{\thepage}
\fancypagestyle{plain}{%
	\fancyhf{} % clear all header and footer fields
	\fancyhead[r]{\leftmark}
	\fancyfoot[c]{PUCP}
	\fancyfoot[r]{\thepage}
	\renewcommand{\headrulewidth}{0.4pt}
	\renewcommand{\footrulewidth}{0.4pt}}
\renewcommand{\headrulewidth}{0.4pt}
\renewcommand{\footrulewidth}{0.4pt}
\renewcommand{\chaptermark}[1]{\markboth{\itshape\chaptername~\thechapter}{}}

\graphicspath{{images/}}

\captionsetup[table]{skip=10pt}
\captionsetup[figure]{skip=10pt}
\captionsetup{justification=centering}

\bibliographystyle{apacite}
\renewcommand\bibname{Bibliografía}

\renewcommand{\BOthers}[1]{et al.\hbox{}}%       et al
\renewcommand{\BOthersPeriod}[1]{et al.\hbox{}}%  et al.



%----------------------------------------------------------------------------------------
%	CONFIGURACION
%----------------------------------------------------------------------------------------
\renewcommand*{\contentsname}{Tabla de contenidos}
\renewcommand*{\listtablename}{Índice de tablas}
\renewcommand*{\listfigurename}{Índice de figuras}
\renewcommand{\baselinestretch}{1.0}
\renewcommand{\appendixname}{Anexos}
\renewcommand{\appendixtocname}{Anexos}
\renewcommand{\appendixpagename}{Anexos}
\renewcommand{\thetable}{\arabic{chapter}.\arabic{table}}
\renewcommand*{\tablename}{Tabla}
\renewcommand*{\chaptername}{Capítulo}
\renewcommand*{\thechapter}{\Roman{chapter}}
\renewcommand{\thesection}{\arabic{chapter}.\arabic{section}}
\renewcommand{\figurename}{Figura}
\renewcommand{\thefigure}{\arabic{chapter}.\arabic{figure}}
\renewcommand{\theequation}{\arabic{chapter}.\arabic{equation}}
\newcommand*\rot{\rotatebox{90}}

